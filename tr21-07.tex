\documentclass{article}
\begin{document}
\title{When to Stop Computing and Start Investing}
\author{Sean R. Aguilar and Olga Kosheleva\\
		University of Texas at El Paso\\500 W. University\\
		El Paso, TX 79968, USA\\sraguilar4@miners.utep.edu, olgak@utep.edu}
\date{}
\maketitle
	
\begin{abstract}
	\textbf{Purpose:} The purpose of the study is to analyze when --
while predicting the future price of a financial instrument -- we
should stop computations and start using this information for the
actual investment.
	
	\textbf{Design/methodology/approach:} We derive the explicit
formulas explaining how the resulting gain depends on the duration
of computations.

	\textbf{Findings:} We provide an algorithm that enables us to
decide the computation time that leads to the largest possible
gain.
	
	\textbf{Originality/value:} To the best of our knowledge, this
is the first solution to the problem. Following our recommendations
will allow investors to select the computation time for which the
resulting gain is the largest possible.
	
	\textbf{Keywords:} Investment; Optimal investment portfolio;
Computation time.
\end{abstract}
	
\section{Formulation of the Problem}

\noindent{\bf Need for predictions.} The price of each stock --
and, in general, of each financial instrument -- reflects the
investors' prediction of how this particular instrument will
change. If the market believes that the company will prosper, the
price of its stock goes up; vice versa, in the market believes that
the company will go into decline, the price of its stock goes down.

Of course, these general market predictions are approximate. As a
result, if some agent can come up with better predictions, this
agent can make money. For example, if the agent's predictions are
higher than the general market's, then it makes sense to buy this
stock and then sell it with profit when its value increases above
the market's expectations.

For this purpose, trading companies use complex prediction
algorithms, ranging from simple numerical models to deep learning;
see, e.g., \cite{Goodfellow 2016}.
\medskip

\noindent{\bf When should we stop computing?} For most prediction
algorithms, the more time we spend on computing, the more accurate
are the predictions. For example, in machine learning algorithms --
including deep learning -- the more data we use for prediction, the
more accurate the predictions; however, the more data we use for prediction, 
the longer the computations
take.

At first glance, it may seem that we should aim for the maximum
possible prediction accuracy. However, if we are predicting tomorrow's
price, and the computations of the most accurate prediction require
more than 24 hours, then this prediction is useless. If computations take 23
hours, then it is also possibly useless, since by then, most of the
increase or decrease in price that we are trying to predict -- and which we plan to use to gain some profit -- would
have already occurred.

So what is the optimal computation time? When should we stop
computing and start using the results of our computations to
actually invest the money?
\medskip

\noindent{\bf What we do in this paper.} In this paper, on the example
of a somewhat simplified model, we provide an answer to this
question. 

We hope that our solution will encourage researchers to solve
this problem in a more general (and more realistic) setting.

\section{Toward Formulating the Problem in Precise Terms}

\noindent{\bf What is our objective.} Investments are risky.
The more risk we tolerate, the more gain we can get. Usually, a
investor selects how much risk he/she can tolerate, and tries to
maximize the expected gain under this restriction on the risk.
\medskip

\noindent{\bf How can we gauge risk?} Investments are risky because
the actual future value of the stock is, in general, different from
our prediction estimate. The future price is not uniquely
determined by today's knowledge and, thus, constitutes what is
called a random variable. There is a natural way to gauge how
different the values of a random variable can be -- we can estimate
it by this variable's {\it standard deviation} $\sigma$ (or,
alternatively, by its variance $V=\sigma^2$).

Thus, a reasonable idea to describe an investor's tolerance to risk
is to have an upper bound $\sigma_0$ on the corresponding standard
deviation per-invested-dollar: $$\sigma\le \sigma_0.$$
\medskip

\noindent{\bf How the tolerated standard deviation depends on
time-to-predicted-event.} If we are predicting the price of the instrument
the next day, the change in the price is usually small, and thus,
the corresponding deviation is small. On the other hand, if we are
predicting next year's price, the changes may be large and thus,
the standard deviation $\sigma$ will be large. In this case, the
bound $\sigma_0$ should be larger.

In general, the larger the time-to-predicted-event, the larger the
corresponding standard deviation, and thus, the larger should be
the corresponding bound. How can we describe the dependence
$\sigma_0(t)$ of the tolerable standard deviation on the
time-to-predicted-event $t$?

Suppose that we have two consequent time intervals, of durations
$t_1$ and $t_2$. Suppose that the investor tolerates the standard
deviation $\sigma_0(t_1)$ for predictions over the first time
interval, and the deviation $\sigma_0(t_2)$ for predictions over
the second time interval. The change in price over the combined
interval of duration $t_1+t_2$ is equal to the sum of the changes
over the intervals of durations $t_1$ and~$t_2$.

For stock market, fluctuations occurring during different time
intervals are largely independent. It is known that the variance of
the sum of two independent random variables is equal to the sum of
their variances; see, e.g., \cite{Sheskin 2011}. Thus, the variance
$\sigma_0^2(t_1+t_2)$ is equal to the sum of the corresponding
variances: $$\sigma_0^2(t_1+t_2)=\sigma_0^2(t_1)+\sigma_0^2(t_2).$$
From this condition, one can easily check that $\sigma_0^2(t)$ is a
linear function of time: $\sigma_0^2(t)=k\cdot t$ for some constant
$t$. Thus, we have $$\sigma_0(t)=s\cdot \sqrt{t},$$ where we
denoted $s\stackrel{\rm def}{=}\sqrt{k}.$
\medskip

\noindent{\bf Let us formulate the problem in precise terms: what we know.}
Suppose that we have a financial instrument -- e.g., a stock or a
portfolio of stocks that closely follows the whole stock marker --
that grows with the average rate $g$. This means that in a year, each
invested dollar will grow by the amount $g$, and during the period
of duration $t$, each invested dollar will grow by the amount
$$b\cdot t.$$

The risk corresponding to this financial instrument is usually larger than what 
a investor can tolerate; so, the investors do not invest all their money
in this instrument. To decrease the risk, they invest some of their money
into this instrument, and the remaining amount of money into 
a no-risk instrument (such as US bonds). If these bonds grow
with the rate $b$, then, in a year, each dollar invested in bonds
will grow by the amount $b$, and during the period of duration $t$,
each invested dollar will grow by the amount $b\cdot t$.

Suppose now that, according to the market's expectations -- which are 
usually obtained simply by analyzing how the price changed in the past -- 
the standard deviation of the change in the per-invested-dollar price of the selected financial
instrument during time $t$ is equal to 
$\sigma_m(t)=m\cdot \sqrt{t}$ for some value $m$; the square-root dependence of this standard
deviation on time can be explained the same way as the dependence
of the risk bound $\sigma_0(t)$ on time. 

Suppose also that our prediction program, when asked
to predict the price of the instrument $t$ moments in the future, after it has been 
computing for time $t_c$, predicts the desired future price with
accuracy $\sigma(t,t_c)$. The longer the
computations last -- i.e., the larger the computation time $t_c$ -- the
more accurate the predictions.

The value $\sigma(t,t_c)$ can be estimated based on the ability of
the software to predict the already observed prices.
\medskip

\noindent{\bf Let us formulate the problem in precise terms: what we want.}
We want to find our at what time $t_c$ we should stop the
computations and use this information for investment -- and what time period $t$ 
should we use for predictions.
\medskip

\noindent{\bf Analysis of the problem.} At first, 
while computations are still going on, to decide which portion $p$ of our money to invest in the
given (risky) financial instrument, we can only use the general market's estimates for this instrument's standard
deviation. 

For each dollar,
we invest the part $p$ in the risky financial instrument, and the
remaining part $1-p$ in the bonds. Then, the expected per-invested-dollar gain is equal
to $$p\cdot g\cdot t+(1-p)\cdot b\cdot t=b\cdot t+p\cdot (g-b)
\cdot t.\eqno{(1)}$$ The portion $p$ is determined by the
condition that this gain should be the largest possible under the
condition that the risk is bounded by value $\sigma_0(t)=s\cdot
\sqrt{t}$. 

We use the market's estimate of the standard deviation of 1 dollar invested in the
instrument. This estimate is $\sigma_m(t)=m\cdot \sqrt{t}$. So, the
standard deviation of a $p$-dollars investment is equal to $p\cdot m\cdot
\sqrt{t}$. Thus, the risk-related constraint takes the form
$$p\cdot m\cdot \sqrt{t}\le s\cdot \sqrt{t},$$ i.e., equivalently,
the form $p\le \displaystyle\frac{s}{m}.$

The larger the portion $p$ invested in the risky financial instrument (such as stock(s)), the larger the
expected return. Thus, under the given constraint, the largest
possible value of the expected return is attained when the
portion $p$ is the largest possible, i.e., when
$p=\displaystyle\frac{s}{m}$. During the time $t_c$ -- while
computations are going on -- we get the gain $$ b\cdot t_c+
\frac{s}{m}\cdot(g-b)\cdot t_c.\eqno{(2)}$$

At the moment $t_c$, we stop the computations and get the 
estimate $\sigma(t,t_c)$. Since this is a more accurate estimate of the instrument's standard deviation
than the market estimate that we previously used, it makes sense to take this new estimate into account and correspondingly re-calculate the portion of money invested
in the instrument. In this case, the risk-related constraint takes the
form $p\cdot \sigma(t,t_c)\le s\cdot \sqrt{t}$, hence $p\le
\displaystyle\frac{s\cdot \sqrt{t}}{\sigma(t,t_c)}$. Thus, we
the new portion $p_{\rm new}$ of money invested in the stock is equal to
$$p_{\rm new}=\frac{s\cdot \sqrt{t}}{\sigma(t,t_c)},$$ and the resulting gain
during the remaining time $t-t_c$ is equal to $$b\cdot (t-t_c)+
\frac{s\cdot \sqrt{t}}{\sigma(t,t_c)}\cdot (g-b)\cdot (t-t_c).\eqno{(3)}$$
By adding up the gains (2) and (3), we get the expression for the
overall gain: $$b\cdot t+s\cdot (g-b)\cdot
\left(\frac{t_c}{m}+
\frac{(t-t_c)\cdot
\sqrt{t}}{\sigma(t,t_c)}\right).\eqno{(4)}$$
In particular, if we divide this amount by the time $t$, we get the
following gain-per-unit-time $G$: $$G=b+s\cdot (g-b)\cdot
\left(\frac{t_c}{t\cdot m}+\frac{t-t_c
}{\sigma(t,t_c)\cdot\sqrt{t}}\right).\eqno{(5)}$$ 
The largest value
of this gain is attained for the values $t_c$ and $t$ for which the
expression $$\frac{t_c}{t\cdot
m}+\frac{t-t_c}{\sigma(t,t_c)\cdot\sqrt{t}}\eqno{(6)}$$ attains the
largest possible value.

\section{Resulting Recommendations}

\noindent{\bf What is given.}
\begin{itemize}
\item We have a
risk-free investment with the expected per-invested-dollar gain~$b$.
\item We know how much risk the investor can tolerate. This
knowledge is described by the value $s$ such that, when investing
for a time $t$, the standard deviation of the change in the per-invested-dollar price of a
portfolio should not exceed $s\cdot \sqrt{t}$.
\item We also have a financial instrument with
the expected per-invested-dollar gain $g$. This expected gain can be determined by analyzing the 
previous prices (and gains) of this instrument.
\item Based on these previous prices and gains, we can also determine the
coefficient $m$ that describes the standard deviation
$\sigma(t)$ of the change of the instrument's per-invested-dollar price during the time
interval $t$: $$\sigma(t)\approx m\cdot\sqrt{t}.$$
\item We also have a prediction algorithm. We can use this algorithm for predicting 
the instrument's price $t$ moments into the future. We can run this algorithm for different amounts $t_c$ of computation time.
Based on applying this
algorithm to the previous data, for each amount $t_c$ of computation time, we estimate the standard deviation
$\sigma(t,t_c)$ of the instrument's future per-invested-dollar price. 
\end{itemize}
\medskip

\noindent{\bf Our recommendation.}
The largest possible gain-per-unit-time occurs for the values $t$
and $t_c$ for which the following expression attains the
largest possible value: $$\frac{t_c}{t\cdot
m}+\frac{t-t_c}{\sigma(t,t_c)\cdot\sqrt{t}}.\eqno{(6)}$$ The resulting gain-per-dollar-and-per-unit-time is equal
to $$G=b+s\cdot (g-b)\cdot
\left(\frac{t_c}{t\cdot m}+\frac{t-t_c
}{\sigma(t,t_c)\cdot\sqrt{t}}\right).\eqno{(5)}$$

\begin{thebibliography}{9}

\bibitem{Goodfellow 2016} I. Goodfellow, Y. Bengio, and A.
Courville, {\it Deep Learning}, MIT Press, Cambridge, Massachusetts,
2016.

\bibitem{Sheskin 2011}
D. J. Sheskin, {\it Handbook of Parametric and Non-Parametric
Statistical Procedures}, Chapman \& Hall/CRC, London, UK, 2011.

\end{thebibliography}
	
\end{document}
